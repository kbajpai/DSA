\documentclass[letterpaper]{book}

\usepackage{tgheros}\renewcommand{\familydefault}{\sfdefault}
\renewcommand{\familydefault}{\sfdefault}

\usepackage{blindtext}
\usepackage{setspace}
\setcounter{secnumdepth}{1}
%\usepackage{color}
\usepackage{xcolor}
\usepackage{tocloft}
\usepackage{listings}
\lstloadlanguages{C,C++,csh,Java}

\renewcommand\cftaftertoctitle{\par\noindent\hrulefill\par\vskip-4.3em}

%%%%Add chapter functionality in article class
\usepackage{hyperref}
\hypersetup{
	colorlinks=true,
	linkcolor=blue,
	filecolor=magenta,
	urlcolor=cyan,
	pdftitle={Introduction to Algorithms},
}
\urlstyle{same}
%%%%

\definecolor{codegreen}{rgb}{0,0.6,0}
\definecolor{codegray}{rgb}{0.5,0.5,0.5}
\definecolor{codepurple}{rgb}{0.58,0,0.82}
\definecolor{backcolour}{rgb}{0.95,0.95,0.92}
\definecolor{cloudwhite}{rgb}{0.9412, 0.9608, 0.8471}
\definecolor{cream}{rgb}{1.0, 0.95, 0.80}
\definecolor{codebg}{rgb}{216, 216, 216}

\lstset{
	language=csh,
	basicstyle=\small\ttfamily,
	numbers=left,
	numberstyle=\tiny,
	numbersep=5pt,
	tabsize=2,
	extendedchars=true,
	breaklines=true,
	frame=b,
	stringstyle=\color{blue}\ttfamily,
	showspaces=false,
	showtabs=false,
	xleftmargin=17pt,
	framexleftmargin=17pt,
	framexrightmargin=5pt,
	framexbottommargin=4pt,
	commentstyle=\color{gray},
	morecomment=[l]{//}, %use comment-line-style!
	morecomment=[s]{/*}{*/}, %for multiline comments
	showstringspaces=false,
	morekeywords={
		abstract, add, alias, allows, and, args, as, ascending, 
		async, await, base, bool, break, by, byte, case, catch, 
		char, checked, class, const, continue, decimal, default, 
		delegate, descending, do, double, dynamic, else, enum, 
		equals, event, explicit, extern, false, file, finally, 
		fixed, float, for, foreach, from, get, global, goto, 
		group, if, implicit, in, init, int, interface, internal, 
		into, is, join, let, lock, long, managed, nameof, namespace, 
		new, nint, not, notnull, nuint, null, object, on, operator, 
		or, orderby, out, override, params, partial, private, protected, 
		public, readonly, record, ref, remove, required, return, sbyte, 
		scoped, sealed, select, set, short, sizeof, stackalloc, static, 
		string, struct, switch, this, throw, true, try, typeof, uint, 
		ulong, unchecked, unmanaged, unsafe, ushort, using, value, var, 
		virtual, void, volatile, when, where, while, with, yield
	},
	keywordstyle=\color{blue},
	identifierstyle=\color{darkgray},	
	backgroundcolor=\color{codebg},
}

\usepackage{caption}
\DeclareCaptionFont{white}{\color{white}}
\DeclareCaptionFormat{listing}{\colorbox{darkgray}{\parbox{\textwidth}{\hspace{15pt}#1#2#3}}}
\captionsetup[lstlisting]{format=listing,labelfont=white,textfont=white, singlelinecheck=false, margin=0pt, font={bf,footnotesize}}

%opening
\title{Introduction to Algorithms}
\author{Kunal Bajpai}

\onehalfspacing
\begin{document}

	\frontmatter

	\maketitle
	\tableofcontents
	\noindent
	\hrulefill

	\chapter*{Preface}
	\addcontentsline{toc}{chapter}{\protect\numberline{}Preface}
	\Blindtext
	\newpage

	\mainmatter
	\part{Foundations}
	\chapter*{Introduction}
	\addcontentsline{toc}{chapter}{\protect\numberline{}Introduction}
	\hrulefill

	\paragraph{Chapter 1}
	Discusses algorithms should be considered as a technology, alongside technologies such as fast hardware, graphical user interfaces, object-oriented systems, and networks.

	\paragraph{Chapter 2}
	\begin{enumerate}
		\item Pseudocode 
		\item Sorting a list of \textit{n} numbers
		\item Insertion Sort, which uses incremental approach
		\item Merge Sort, which uses recursive approach known as \textbf{divide-and-conquer}
		\item Running times between these two algorithms \& develop useful notations to express them 
	\end{enumerate}
	
	\paragraph{Chapter 3} 
	Precisely defines \textbf{Asymptotic Notation} \& mathematical concepts to express running times of algorithms 

	\paragraph{Chapter 4} 
	\Blindtext	

	\chapter{The Role of Algorithms in Computing}
	\section{Algorithms}

	\begin{lstlisting}[language={[Sharp]C}, caption={Insertion Sort}, label={Script}]
	//Here's the comment
	public static int[] InsertionSort(int[] a) {
		for (var i = 1; i < a.Length; i++)
		{
			var n = a[i];
			var j = i;
			while (j > 0 && n < a[j - 1])
			{
				a[j] = a[j - 1];
				j--;
			}

			a[j] = n;
		}

		return a;
	}
	\end{lstlisting}

	\Blindtext
	\newpage

	\section{Algorithms as a technology}
	\Blindtext
	\newpage

	\chapter{Getting Started}
	\section{Insertion sort}
	\Blindtext
	\newpage

	\section{Analyzing algorithms}
	\Blindtext
	\newpage

	\section{Designing algorithms}
	\Blindtext
	\newpage

	\chapter{Growth of Functions}
	\section{Asymptotic notation}
	\Blindtext
	\newpage

	\section{Standard notations and common functions}
	\Blindtext
	\newpage

	\part{Appendix: Mathematical Background}

	\appendix
	\chapter{Bibliography}
	\section{References}
	\Blindtext
	\newpage

	\chapter*{Introduction}
	\addcontentsline{toc}{chapter}{\protect\numberline{}Introduction}

	\chapter*{Summations}
	\addcontentsline{toc}{chapter}{\protect\numberline{}Summations}
	\section{Summation formulas and properties}
	\section{Bounding summations}
	\Blindtext
	\newpage

	\backmatter
	\chapter{Bibliography}
	\section{References}
	\Blindtext
	\newpage

\end{document}