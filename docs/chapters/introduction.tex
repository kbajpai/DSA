\chapter{Introduction}
\label{introduction}

The course is split up into 12 chapters. The bulk of the content is within chapters 1 - 10, each of which focuses on a data structure or algorithm:

\begin{enumerate}
	\item{} Arrays and Strings
	\item{} Hashmaps and Sets
	\item{} Linked Lists
	\item{} Stacks and Queues
	\item{} Trees and Graphs
	\item{} Heaps
	\item{} Greedy Algorithms
	\item{} Binary Search
	\item{} Backtracking
	\item{} Dynamic Programming
\end{enumerate}

\textbf{These are the most important and most common data structures and algorithms for coding interviews.}

\textbf{Appendix A} contains several useful tools that can be used moving forward. There are code templates for all common patterns, cheatsheets regarding time and space complexities, and a flowchart that can be used as a general guideline when trying to figure out what data structure or algorithm should be used.

\textbf{Appendix B} contains few more techniques that aren't common/broad enough to warrant their own chapter.

\pagebreak

{\huge Introduction to Big-O}

{\large What is an algorithm?}

An algorithm can be seen as a recipe for a computer to follow. It's a set of instructions that a computer will follow step-by-step to solve a problem.

There are some important requirements for algorithms in the context of LeetCode:

\begin{enumerate}
	\item{} Algorithms should be \textbf{deterministic}. Given the same input, the algorithm should \textbf{always} produce the same \textbf{output}.
	\item{} The algorithm should be correct for any arbitrary valid input.
\end{enumerate}

{\large Big O}

Big O is a notation used to describe the computational complexity of an algorithm. The computational complexity of an algorithm is split into two parts:

\begin{enumerate}
	\item{} Time Complexity: The time complexity of an algorithm is the amount of time the algorithm needs to run relative to the input size.
	\item{} Space Complexity: The space complexity of an algorithm is the amount of memory allocated by the algorithm when run relative to the input size.
\end{enumerate}
